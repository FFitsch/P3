\documentclass[titlepage]{article}
\usepackage{fancyhdr}
\usepackage[margin=1.2in]{geometry}
\usepackage[title]{appendix}

\usepackage[utf8]{inputenc}
\usepackage[english]{babel}
\usepackage{csquotes}

\usepackage[backend=biber,style=chem-acs,sorting=ynt,url=true,doi=true]{biblatex}
\addbibresource{P3.bib}

\usepackage[breaklinks]{hyperref}

\usepackage{graphicx}
\usepackage{float}
\usepackage[tableposition=top, justification=centering]{caption}
\usepackage{multirow}

\usepackage{derivative}

\usepackage{array,tabularx,calc}
\newlength{\conditionwd}
\newenvironment{conditions}[1][where:]{
        #1\tabularx{\linewidth-\widthof{#1}}[t]
        {>{$}l<{$} @{${}={}$} X@{}}}
  {\endtabularx\\[\belowdisplayskip]}


% Title Page -----------------------------------------------------------
\title{Protocol \\ Physikal.-chem. Praktikum 165.008 \\ P3 - Franck-Hertz Experiment}
\author{Group F\\Jonas Adamer (12225913)\\Florian Fitsch (12218283)\\Leonhard Ritt (12208881)}
\date{Date of experiment: 2024/10/28\\Date of submission: 2024/11/12}

\pagestyle{fancy}
\fancyhf{}
\fancyhead[R]{\thepage}
\fancyfoot[L]{P3: Franck-Hertz Experiment}
\fancyfoot[C]{Group F}
\fancyfoot[R]{Adamer, Fitsch, Ritt}

\begin{document}

% TITLE PAGE
\maketitle
\thispagestyle{empty}

% TABLE OF CONTENTS
\newpage
\tableofcontents
\thispagestyle{empty}

\addtocounter{page}{-1}

\newpage
\section{Objective}
In this assignment, the Franck-Hertz experiment is carried out using a neon- as well as a nercury-filled tube. While the neon-tube is run at room temperature, the ,mercury tube is run at Temperatures of 150 as well as 180~°C. The available experimental parameters (suction voltage as well as retarding voltage) are tuned in order to attain several clear maxima and minima in the Franck-Hertz curve, before measurements are taken in order to determine the excitation energies of the studied elements.

\section{Experiment}
\subsection{Theory}
The Franck-Hertz experiment is used to show the quantization of electronic states in atoms. The core piece of its experimental setup is a sealed tube, which is filled with an atomic gas (such as neon or vaporized mercury). Along its length there are four components:
%
\begin{enumerate}
    \itemsep0em
    \item A cathode (C), which can be heated by applying a current across it.
    \item Two metal wire nets (N\textsubscript{1} and N\textsubscript{2}).
    \item An anode (A).
\end{enumerate}

By heating up the cathode K and applying a potential (suction voltage U\textsubscript{1}) between it and N\textsubscript{1}, electrons are emitted from K and drawn towards N\textsubscript{1}. Another potential is applied between N\textsubscript{1} and N\textsubscript{2}. This is the accelerating voltage U\textsubscript{2}, which accelerates the previously emitted electrons over a fixed distance between the nets. A third potential (retarding voltage U\textsubscript{3}) is applied in opposite polarity to U\textsubscript{1} and U\textsubscript{2} between N\textsubscript{2} and A, which decelerates the electrons, only allowing those with a sufficient kinetic energy to reach A.

During their travel between N\textsubscript{1} and N\textsubscript{2}, electrons of sufficient kinetic energy are able to inelastically collide with the atoms filling the tube, causing excitation of one of the atom's valence electrons into a higher electronic state. This lowers the free electrons' kinetic energy by the value of the excitation energy \(\Delta\)E of the respective element.

By recording the anode current I\textsubscript{A} while continuously increasing U\textsubscript{2}, Franck-Hertz curves are recorded, in which the following phenomena can be observed:

\begin{enumerate}
    \itemsep0em
    \item At first, almost no electrons have the required kinetic energy to pass E\textsubscript{3} and reach A, which results in a low I\textsubscript{A}.
    \item As U\textsubscript{2} is raised, more electrons reach A, increasing I\textsubscript{A}.
    \item Once electrons start gaining sufficient kinetic energy between N\textsubscript{1} and N\textsubscript{2}, they cause excitations of the atoms in the tube. Thus they lose kinetic energy, and I\textsubscript{A} decreases.
    \item With even higher U\textsubscript{2}, electrons are able to gain enough kinetic energy after causing an excitation to reach the anode and thus I\textsubscript{A} increases once more.
    \item Eventually, electrons are able to excite a second atom before reaching N\textsubscript{2}, which results in a decrease of I\textsubscript{A}.
    \item The previous steps repeat, causing an oscillation of I\textsubscript{A}.
\end{enumerate}

By measuring the difference in U\textsubscript{2} between two minima or maxima of I\textsubscript{A} in the recorded curve, the excitation energy \(\Delta\)E of the atoms within the tube can be determined. This proves the quantization of electronic states within atoms, since, during every excitation caused by the electrons, the same quantized amount of energy (the energy difference between the ground state and the first excited state) is given off to the atom.

\subsection{Practical Realization}
In order to start the experiment, the neon-filled tube was connected to the Franck-Hertz control device. Using its various knobs, the voltages U\textsubscript{1} (suction voltage between cathode and first grid) and U\textsubscript{3} (retarding voltage between the anode and second grid) were set to the initial values presented in Table \ref{tb_operational_parameters}. These values were then tuned by setting the control device to quickly sweep the entire available range for U\textsubscript{2} (accelerating voltage between the two grids), while an oscilloscope was used to view the Franck-Hertz curve, in order to get clear and numerous peaks and troughs.
%
\begin{table}[H]
  \centering
  \caption{Initial and tuned operational values of the conducted Franck-Hertz experiments.}
  \label{tb_operational_parameters}
  \begin{tabular}{
    |>{\centering\arraybackslash}m{0.08\textwidth}
    |>{\centering\arraybackslash}m{0.125\textwidth}
    |>{\centering\arraybackslash}m{0.125\textwidth}
    |>{\centering\arraybackslash}m{0.125\textwidth}
    |>{\centering\arraybackslash}m{0.125\textwidth}
    |>{\centering\arraybackslash}m{0.125\textwidth}
    |>{\centering\arraybackslash}m{0.125\textwidth}
    |}
      \hline
       & \textbf{Ne\newline initial} & \textbf{Ne\newline tuned} & \textbf{Hg, 150~°C\newline initial} & \textbf{Hg, 150~°C\newline tuned} & \textbf{Hg, 180~°C\newline initial} & \textbf{Hg, 180~°C\newline tuned}
      \\
      \hline
      \textbf{U\textsubscript{1} [V]} & 0.50 & 0.54 & 5.00 & 5.00 & 5.00 & 5.31
      \\
      \hline
      \textbf{U\textsubscript{3} [V]} & 7.00 & 8.98 & 1.50 & 4.08 & 1.50 & 0.68
      \\
      \hline
  \end{tabular}
\end{table}
%
\noindent After the ideal parameters were determined, an X/Y-plotter was connected to the control device, with U\textsubscript{2} recorded on the x-axis and the anode current being recorded on the y-axis. Measurements were then plotted to paper using the control device's sweep-function (sweeping over the available range of U\textsubscript{2} values), both in the plotter's CAL-mode, in which one unit on the paper corresponds to a fixed input current, as well as the VAR-mode, in which the plot was manually scaled to best fit the paper.

Following the experiments using the neon tube, the mercury tube was connected instead and heated up by setting the desired temperature on the control device. The experimental steps described above were then repeated at tube temperatures of 150 and 180~°C.

\newpage
\section{Results}
\subsection{Determination of Excitation Energy for Neon}
Figures \ref{fig_ne_CAL} and \ref{fig_ne_VAR} show the recorded Franck-Hertz curves of neon in the CAL- and VAR-modes of the X/Y-plotter respectively. With the starting point of the plot set as (0,0), the x-values of the peaks and troughs of the curves were measured and recorded in Table \ref{tb_ne_peaks_troughs}.
%
\begin{figure}[H]
    \centering
    \includegraphics[width=0.45\textwidth]{Figures/Ne_CAL.png}
    \caption{Recorded Franck-Hertz curve of neon using the X/Y-plotter's CAL-mode at a horizontal resolution of 1~V/mm.}
    \label{fig_ne_CAL}
\end{figure}
%
\begin{figure}[H]
    \centering
    \includegraphics[width=\textwidth]{Figures/Ne_VAR.png}
    \caption{Recorded Franck-Hertz curve of neon using the X/Y-plotter's VAR-mode.}
    \label{fig_ne_VAR}
\end{figure}
%
\begin{table}[H]
  \centering
  \caption{X-values of measured peaks and throughs in the CAL- and VAR-mode Ne-plots}
  \label{tb_ne_peaks_troughs}
  \begin{tabular}{
    |>{\centering\arraybackslash}m{0.125\textwidth}
    |>{\centering\arraybackslash}m{0.125\textwidth}
    |>{\centering\arraybackslash}m{0.125\textwidth}
    |}
      \hline
       & \textbf{X-value\newline CAL [mm]} & \textbf{X-value\newline VAR [mm]}
      \\
      \hline
      \textbf{Peak 1} & 19 & 63
      \\
      \hline
      \textbf{Peak 2} & 36 & 120
      \\
      \hline
      \textbf{Peak 3} & 56 & 183
      \\
      \hline
      \textbf{Trough 1} & 27 & 88
      \\
      \hline
      \textbf{Trough 2} & 46 & 152
      \\
      \hline
      \textbf{Trough 3} & 62 & 205
      \\
      \hline
  \end{tabular}
\end{table}
%
\noindent In order to be able to use the more precise VAR-values for further calculations, the reduction factor of the VAR-scaling has to be calculated. This is done by dividing a specific distance in CAL-mode from the equivalent distance in VAR-mode. For the purposes of this experiment, the distance in X-direction between the first and last peak is used.
%
\begin{equation} \label{eq_reduction_factor}
    R = \frac{X_{cal,2} - X_{cal,1}}{X_{var,2} - X_{var,1}}
\end{equation}
\begin{conditions}
    R & Reduction factor \\
    X_{cal,1}, X_{cal,2} & X-coordinates of specified points on the CAL-plot \\
    X_{var,1}, X_{var,2} & X-coordinates of specified points on the VAR-plot
\end{conditions}

\noindent Using the values of Peaks 1 and 3 in Table \ref{tb_ne_peaks_troughs}, a reduction factor of 0.3083 is determined. Since, during the CAL-measurement, the X/Y-plotter was set to record one volt of input voltage as one millimeter on the paper, the distance of two neighboring peaks or troughs on the VAR-graph only needs to be multiplied with the reduction factor in order to calculate the excitation energy of the studied element in electron-volts (eV):
%
\begin{equation} \label{eq_excitation_energy}
    \adif E = (X_{var,n} - X_{var,n-1}) \cdot R
\end{equation}
\begin{conditions}
    X_{cal,n-1}, X_{cal,n} & X-coordinates of neighboring peaks or troughs in the VAR-Plot
\end{conditions}

\noindent This calculation was done using all peak and trough pairs and its results are shown in Table \ref{tb_ne_excitation_energies}.
%
\begin{table}[H]
    \centering
    \caption{Calculated excitation energies of neon from all measured peak and trough pairs.}
    \label{tb_ne_excitation_energies}
    \begin{tabular}{
      |>{\centering\arraybackslash}m{0.15\textwidth}
      |>{\centering\arraybackslash}m{0.15\textwidth}
      |}
        \hline
         & \textbf{\(\Delta\)E [eV]}
        \\
        \hline
        \textbf{Peaks 1,2} & 17.58
        \\
        \hline
        \textbf{Peaks 2,3} & 19.43
        \\
        \hline
        \textbf{Troughs 1,2} & 19.73
        \\
        \hline
        \textbf{Troughs 2,3} & 16.34
        \\
        \hline
    \end{tabular}
\end{table}
%
\noindent Using Equation \ref{eq_arithmetic_mean}, the arithmetic mean of these values is calculated. The result of this is the determined activation energy.
%
\begin{equation} \label{eq_arithmetic_mean}
    \overline{x} = \frac{1}{n} \cdot \sum_{i=1}^{n} x_i
\end{equation}
\begin{conditions}
    \overline{x} & Arithmetic mean of taken measurements \\
    n & Number of measurements \\
    x_i & Value of measurement i
\end{conditions}

\noindent In order to calculate the total error of this result, two different error sources need to be evaluated.

Firstly, the uncertainty in voltage readings, resulting from the limited resolution of the used plot paper, needs to be taken into account. This is simply double the lowest measurable voltage (double since the voltage difference is calculated from two single measurements). This in turn is equal to the resolution of the paper (1~mm) multiplied by the reduction factor R.
%
\begin{equation} \label{eq_resolution_uncertainty}
    u_{res} = 2 \cdot R \cdot r
\end{equation}
\begin{conditions}
    u_{res} & Uncertainty resulting from measurement resolution \\
    R & Reduction factor \\
    r & Resolution of the paper
\end{conditions}

\noindent Secondly, the random uncertainty of the taken measurements can be calculated as the standard error of the mean:
%
\begin{equation} \label{eq_random_uncertainty}
    u_{sem} = \frac{1}{\sqrt{n}} \cdot \sqrt{\frac{1}{n-1} \cdot \sum_{i=1}^{n} (x_i - \overline{x})^2}
\end{equation}
\begin{conditions}
    u_{sem} & Standard error of the mean \\
    n & Number of measurements \\
    x_i & Value of measurement i \\
    \overline{x} & Arithmetic mean of measurements
\end{conditions}

\noindent From these two values, the total uncertainty can then be calculated:
%
\begin{equation} \label{eq_total_uncertainty}
    u = \sqrt{u_{res}^2 + u_{sem}^2}
\end{equation}
\begin{conditions}
    u & Total uncertainty \\
\end{conditions}

\noindent Using the equations above, the excitation energy of neon \(\adif E_{Ne}\) is determined as \textbf{(18.26~\(\pm\)~1.01)~eV}.

\subsection{Determination of Excitation Energy for Mercury} \label{ssec_neon_results}
Figures \ref{fig_hg_150_CAL} through \ref{fig_hg_180_VAR} show the recorded Franck-Hertz curves of mercury in the CAL- and VAR-modes of the X/Y-plotter respectively. With the starting point of the plot set as (0,0), the x-values of the peaks and troughs of the curves were measured and recorded in Table \ref{tb_hg_peaks_troughs}.
%
\begin{figure}[H]
    \centering
    \includegraphics[width=0.3\textwidth]{Figures/Hg_150_CAL.png}
    \caption{Recorded Franck-Hertz curve of mercury at 150~°C using the X/Y-plotter's CAL-mode at a horizontal resolution of 1~V/mm.}
    \label{fig_hg_150_CAL}
\end{figure}
%
\begin{figure}[H]
    \centering
    \includegraphics[width=\textwidth]{Figures/Hg_150_VAR.png}
    \caption{Recorded Franck-Hertz curve of mercury at 150~°C using the X/Y-plotter's VAR-mode.}
    \label{fig_hg_150_VAR}
\end{figure}
%
\begin{figure}[H]
    \centering
    \includegraphics[width=\textwidth]{Figures/Hg_180_CAL.png}
    \caption{Recorded Franck-Hertz curves of neon using the X/Y-plotter's CAL-mode at a horizontal resolution of 1~V/mm. Due to erroneous peaks occuring during some measurements, multiple curves were recorded.}
    \label{fig_hg_180_CAL}
\end{figure}
%
\begin{figure}[H]
    \centering
    \includegraphics[width=\textwidth]{Figures/Hg_180_VAR.png}
    \caption{Recorded Franck-Hertz curve of neon using the X/Y-plotter's VAR-mode.}
    \label{fig_hg_180_VAR}
\end{figure}
%
\begin{table}[H]
    \centering
    \caption{X-values of measured peaks and throughs in the CAL- and VAR-mode Hg-plots at both 150~°C and 180~°C}
    \label{tb_hg_peaks_troughs}
    \begin{tabular}{
      |>{\centering\arraybackslash}m{0.15\textwidth}
      |>{\centering\arraybackslash}m{0.15\textwidth}
      |>{\centering\arraybackslash}m{0.15\textwidth}
      |>{\centering\arraybackslash}m{0.15\textwidth}
      |>{\centering\arraybackslash}m{0.15\textwidth}
      |}
        \hline
         & \textbf{X-value CAL\newline 150~°C [mm]} & \textbf{X-value VAR\newline 150~°C [mm]} & \textbf{X-value CAL\newline 180~°C [mm]} & \textbf{X-value VAR\newline 180~°C [mm]}
        \\
        \hline
        \textbf{Peak 1} & 4 & 22 & 6 & 48
        \\
        \hline
        \textbf{Peak 2} & 8 & 59 & 11 & 86
        \\
        \hline
        \textbf{Peak 3} & 13 & 100 & 16 & 125
        \\
        \hline
        \textbf{Peak 4} & 19 & 141 & 21 & 163
        \\
        \hline
        \textbf{Peak 5} & 24 & 184 & 27 & 204
        \\
        \hline
        \textbf{Peak 6} & 30 & 233 & 32 & 243
        \\
        \hline
        \textbf{Trough 1} & 6 & 46 & 8 & 61
        \\
        \hline
        \textbf{Trough 2} & 11 & 84 & 13 & 101
        \\
        \hline
        \textbf{Trough 3} & 16 & 123 & 18 & 141
        \\
        \hline
        \textbf{Trough 4} & 21 & 163 & 23 & 181
        \\
        \hline
        \textbf{Trough 5} & 26 & 204 & 28 & 221
        \\
        \hline
    \end{tabular}
\end{table}
%
\noindent Using Equation \ref{eq_reduction_factor}, reduction factors for both temperatures were calculated using the first and last peaks. For 150 and 180~°C, these are 0.1232 and 0.1333, respectively. Just like in the neon experiments, a scale of one volt per millimeter was used and as such the reduced VAR-values can be directly read as Voltages.

Using Equation \ref{eq_excitation_energy}, the excitation energies of each peak and trough pair was calculated and recorded in Table \ref{tb_hg_excitation_energies}
%
\begin{table}[H]
    \centering
    \caption{Calculated excitation energies of mercury from all measured peak and trough pairs.}
    \label{tb_hg_excitation_energies}
    \begin{tabular}{
      |c
      |>{\centering\arraybackslash}m{0.15\textwidth}
      |>{\centering\arraybackslash}m{0.15\textwidth}
      |}
        \hline
         & & \textbf{\(\Delta\)E [eV]}
        \\
        \hline
        \multirow{9}{*}{\rotatebox[origin=c]{90}{\textbf{150~°C}}} & \textbf{Peaks 1,2} & 4.56
        \\
        \cline{2-3}
         & \textbf{Peaks 2,3} & 5.05
        \\
        \cline{2-3}
         & \textbf{Peaks 3,4} & 5.05
        \\
        \cline{2-3}
         & \textbf{Peaks 4,5} & 5.30
        \\
        \cline{2-3}
         & \textbf{Peaks 5,6} & 6.04
        \\
        \cline{2-3}
         & \textbf{Troughs 1,2} & 4.68
        \\
        \cline{2-3}
         & \textbf{Troughs 2,3} & 4.81
        \\
        \cline{2-3}
         & \textbf{Troughs 3,4} & 4.93
        \\
        \cline{2-3}
         & \textbf{Troughs 4,5} & 5.05
        \\
        \hline
        \multirow{9}{*}{\rotatebox[origin=c]{90}{\textbf{180~°C}}} & \textbf{Peaks 1,2} & 5.07
        \\
        \cline{2-3}
         & \textbf{Peaks 2,3} & 5.20
        \\
        \cline{2-3}
         & \textbf{Peaks 3,4} & 5.07
        \\
        \cline{2-3}
         & \textbf{Peaks 4,5} & 5.47
        \\
        \cline{2-3}
         & \textbf{Peaks 5,6} & 5.20
        \\
        \cline{2-3}
         & \textbf{Troughs 1,2} & 5.33
        \\
        \cline{2-3}
         & \textbf{Troughs 2,3} & 5.33
        \\
        \cline{2-3}
         & \textbf{Troughs 3,4} & 5.33
        \\
        \cline{2-3}
         & \textbf{Troughs 4,5} & 5.33
        \\
        \hline
    \end{tabular}
\end{table}
%
\noindent Using Equations \ref{eq_arithmetic_mean}\textendash\ref{eq_total_uncertainty}, the arithmetic means as well as the uncertainties of the excitation energies at 150 and 180~°C are calculated.

The thusly calculated excitation energies are determined as \textbf{(5.05~\(\pm\)~0.29)~eV} at 150~°C and \textbf{(5.26~\(\pm\)~0.27)~eV} at 180~°C.

\newpage
\section{Disucssion}
The determined excitation energy of neon at (18.26~\(\pm\)~1.01)~eV differs significantly from the literature value for the lowest excitation of neon at 16.62~eV \autocite{Ne_Energies}. This is due to the fact that, after acceleration of an electron to a kinetic energy high enough to inelastically collide and cause an excitation, the electron continues to travel for the length of its mean free path, before doing so. During this time, it is further accelerated by the applied electric field and, upon eventual collision, causes an excitation to a higher energy state than the cited value \autocite{New_Features_of_FH_Exp}.

By instead comparing to the energy of higher excitation states, one can see that the excitation energy of an electron from a 2p to a 3p orbital (18.38\textendash18.97~eV) lies well within the range of uncertainty of the determined energy value \autocite{Ne_Energies}.

Looking at the determined excitation energies for mercury of (5.05~\(\pm\)~0.29)~eV at 150~°C and (5.26~\(\pm\)~0.27)~eV at 180~°C, one can similarly see that these are higher than the lowest excitation of Hg at 4.67~eV, while more closely matching the higher excitation levels at 4.89 and 5.46~eV\autocite{Hg_Energies}. The three discussed values for mercury all represent an excitation from a 6s to a 6p orbital - they differ however in spin and angular momentum quantum numbers.

Comparing the determined excitation energies at different temperatures, a higher excitation energy is observed at 180~°C than at 150~°C. This contradicts with observations in literature, where a decrease in peak/trough distance (and thus determined excitation energy) is observed with an increase in temperature, due to the electrons' shorter mean free path at higher temperatures \autocite{New_Features_of_FH_Exp}.

Another expected trend for all measurements is an increasing peak/trough distance with an increase in acceleration voltage U\textsubscript{2}. This, again, is due to the additional travel of electrons over the mean free path after reaching sufficient energy to cause an excitation. For each excitation, this distance is effectively 'lost', before an electron can again be accelerated in preparation for another excitation - as such, with more inelastic collisions, the effective tube length, over which the supplied voltage has to accelerate the electrons, is shortened, meaning a larger potential is required \autocite{New_Features_of_FH_Exp}. While with neon - presumably due to the low amount of data points - this phenomenon can not be seen consistently (while between the two evaluted peak pairs, an increase in peak distance is visible, the opposite is seen for the observed troughs), three of four measurement series of mercury follow the expected trend (while the troughs in the Franck-Hertz curve at 180~°C show a static trough distance). The linear regression curves of peak/trough distance with n (peak number from left to right) are shown in Appendix \ref{sec_apx_lin_reg}.

\section{Conclusion}
Through application of the Franck-Hertz experiment, the quantum nature of atoms was observed. The functional parameters of the experiment were tuned in order to achieve clear Franck-Hertz curves of neon and mercury, from which excitation energies were determined. Additional trends and featuers of the experiment, such as temperature- and acceleration-voltage-dependency of the observed energy differences were discussed using relevant literature and compared to the experimental values.

\newpage
\printbibliography

\newpage
\begin{appendices}
\section{Linear Regression of Peak/Trough Distance against Peak Number} \label{sec_apx_lin_reg}
\begin{figure}[H]
    \centering
    \includegraphics[width=0.8\textwidth]{Figures/Hg_150_Lin_Reg.png}
    \caption{Linear regression curves of peak (blue) and trough (orange) distance in eV against peak number n for mercury at 150~°C.}
    \label{fig_hg_150_Lin_Reg}
\end{figure}
%
\begin{figure}[H]
    \centering
    \includegraphics[width=0.8\textwidth]{Figures/Hg_180_Lin_Reg.png}
    \caption{Linear regression curves of peak (blue) and trough (orange) distance in eV against peak number n for mercury at 180~°C.}
    \label{fig_hg_180_Lin_Reg}
\end{figure}
\end{appendices}
\end{document}